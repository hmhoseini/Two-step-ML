\documentclass[DIV=13,10pt]{scrartcl}
\usepackage[utf8]{inputenc}
\usepackage[margin=2cm, footskip=.8cm]{geometry} % reduce page margin

\usepackage{pdflscape} % to switch to landscape mode
% \usepackage[ngerman]{babel}
% \usepackage[ngerman]{babel}
%%%%% Images %%%%%
\usepackage{graphicx}
% \usepackage{svg} % allow using .svg images
\usepackage{float} % allows abritrary figure positions using option [H]
\usepackage{wrapfig} % allows text wrapping around figures
\usepackage{subfig} % images next to each other
\usepackage{caption}                 % sans serif image captions #part 1
\captionsetup{font=sf, labelfont=bf} % sans serif image captions #part 2

\usepackage{url}
\usepackage[table]{xcolor}
% \usepackage{subfigure}
\usepackage{enumerate}

\usepackage{centernot} % um Dinge zentriert durchzustreichen
\usepackage{paralist}

\usepackage{amssymb}
    \let\oldemptyset\emptyset
    \let\emptyset\varnothing
\usepackage{mathtools} %includes amsmath
\usepackage{amsthm}
\usepackage{mleftright}
\usepackage{mathrsfs} %für Befehl mathscr
\usepackage{commath}
\usepackage{relsize} %für mathsmaller
\usepackage{xfrac}
\usepackage{nicefrac}
\usepackage{mathdots} %für iddots
\usepackage{csquotes} % for quoting and bibliography references

%%% generate hyperlinks in the document
\usepackage{hyperref}
\hypersetup{
    colorlinks = true,
    urlcolor = blue, % color of online references
    linkcolor = mydarkblue} % color of local references

\usepackage[backend=bibtex, style=alphabetic]{biblatex}   % bibliography
\addbibresource{sources.bib}
\renewcommand*{\bibfont}{\sffamily} % change bibliograpgy font 


\allowdisplaybreaks % allow page breaks in align
\setlength{\parindent}{0mm} % remove indentation at the start of new paragraphs

%%%%% Code Listings %%%%%
% provides commands for including code (python, latex, ...)
\usepackage{listings}
\definecolor{keywords}{RGB}{255,127,33}  %orange
\definecolor{comments}{RGB}{150,150,150} %medium grey
\definecolor{blue}{RGB}{0,0,255}
\definecolor{fav_blue}{RGB}{85,136,255}      %blue
\definecolor{green}{RGB}{0,168,0}        %green
\definecolor{grey}{RGB}{50,50,50}        %dark grey
\definecolor{light_grey}{RGB}{245,245,245}%light grey
\lstset{language=Python, 
        basicstyle = \ttfamily\small, 
        keywordstyle = \color{keywords},
        commentstyle = \color{comments},
        stringstyle = \color{green},
        showstringspaces = false,
        identifierstyle = \color{grey},
        backgroundcolor = \color{light_grey},
        xleftmargin = 10pt,  %text margin left
        xrightmargin = 10pt, %text margin right
        framexleftmargin = 5pt,  %frame margin left
        framexrightmargin = 5pt, %frame margin right
        numbers = left
        }

%%% the following produces a new environment like array but with automatic page breaks
\usepackage{array,longtable}
\usepackage{arydshln} % allow dashed table lines
\setlength{\dashlinegap}{2pt}
\setlength{\dashlinedash}{2pt}
\newcolumntype{C}{>{\(\displaystyle}c<{\)}}  % automatic math mode, centered
\newcolumntype{R}{>{\(\displaystyle}r<{\)}}
\newcolumntype{L}{>{\(\displaystyle}l<{\)}}
\newcolumntype{q}{>{\(}c<{\)}}  % automatic math mode, centered
\newcolumntype{s}{>{\(}r<{\)}}
\newcolumntype{e}{>{\(}l<{\)}}
\setlength\tabcolsep{5pt}     % match value of \arraycolsep
\renewcommand{\arraystretch}{1.2} % increased distance between lines in table and align environments

\usepackage{chngcntr} % make table numbers depend on section
\counterwithin{table}{section}
\usepackage{multicol}

%%% packages for pseudocode:
\usepackage{algorithm}
\usepackage{algorithmicx}
\usepackage{algpseudocode}

%%% TODO explain packages
% \usepackage{fancyhdr}
% \usepackage{lastpage}
% \pagestyle{fancy}
% \fancyhf{}
% \chead{Sebastian Jost}

%%%%% Math commands %%%%%

\DeclarePairedDelimiterX{\nor}[1]{\lVert}{\rVert}{#1}
\newcommand{\C}{\mathbb{C}}
\newcommand{\R}{\mathbb{R}}
\newcommand{\Q}{\mathbb{Q}}
\newcommand{\Z}{\mathbb{Z}}
\newcommand{\N}{\mathbb{N}}
\newcommand{\E}{\mathbb{E}}
\newcommand{\Pb}{\mathbb{P}}
\usepackage{bbm} % für mathbbm
\newcommand{\indf}[1]{\mathbbm{1}_{#1}} % Indikator Funktion

\newcommand{\bci}[1][k]{\bigcup_{#1=1}^{\infty}}
\newcommand{\too}[0]{\longrightarrow}
\newcommand{\limn}[0]{\lim_{n\to \infty}}

\newcommand{\eps}[0]{\varepsilon}
\newcommand{\cnot}[0]{\centernot} % \cnot as short version of \centernot
\newcommand{\ot}[0]{\leftarrow}
\newcommand\numberthis{\addtocounter{equation}{1}\tag{\theequation}} % add numbering manually

%%% Verteilungen
\DeclareMathOperator{\Exp}{Exp}
\DeclareMathOperator{\Geo}{Geo}
\DeclareMathOperator{\Bin}{Bin}
\DeclareMathOperator{\id}{id}

%%% Varianz und Kovarianz
\DeclareMathOperator{\Var}{Var}
\DeclareMathOperator{\Cov}{Cov}

\DeclareMathOperator{\diag}{diag}

\include{colors}

% \title{Vergleich von Architekturen und Parametern neuronaler Netzwerke}
\title{Result tables for experiments with cAdam}
\author{Sebastian Jost}
\date{\today}

\let\oldsubsubsection=\subsubsection
\renewcommand{\subsubsection}{%
  \filbreak
  \oldsubsubsection
}
% use bibliography file
\bibliography{sources}
% \bibliographystyle{abbrv}

\begin{document}
\sffamily
% \input{title_page}
\tableofcontents
% \newpage
\begin{landscape}



%%%%%%%%%%%%%%%%%%%%%%%%%%%%%%%%%%%%%%%%%%%%%%%%%%%%%%%%%%%%%%%%%%%%%%%
\section{new experiments: chemReg with cAdam}
\subsection{new experiments:chemReg with cAdam sorted by val MAE}
\begin{longtable}{|l|>{\columncolor{bestColumnColor}}l|l|l|l|l|}
\hline
\textbf{parameter name} & \multicolumn{5}{c|}{\textbf{best values}} \\
\hline
\textit{'final val loss mae avg'} (avg) &  0.1116 & 0.11208 & 0.11416 & 0.11493 & 0.11534 \\
test loss mae avg        & 0.11389 & 0.11298 & 0.11493 & 0.11445 & 0.11539 \\
test loss avg            & 0.012971 & 0.012764 & 0.0065903 & 0.0065354 & 0.0066426 \\
training time avg        & 44.379  & 566.85  & 577.92  & 80.447  & 44.17   \\
neurons per layer        & (50, 10) & (50, 10) & (50, 10) & (30, 30, 10) & (40, 20) \\
{\color{equalParamColor} activation functions } & {\color{equalParamColor} sigmoid } & {\color{equalParamColor} sigmoid } & {\color{equalParamColor} sigmoid } & {\color{equalParamColor} sigmoid } & {\color{equalParamColor} sigmoid } \\
{\color{equalParamColor} last activation function } & {\color{equalParamColor} linear } & {\color{equalParamColor} linear } & {\color{equalParamColor} linear } & {\color{equalParamColor} linear } & {\color{equalParamColor} linear } \\
loss function            & MSE     & MSE     & log cosh & log cosh & log cosh \\
{\color{equalParamColor} training data percentage } & {\color{equalParamColor} 1 } & {\color{equalParamColor} 1 } & {\color{equalParamColor} 1 } & {\color{equalParamColor} 1 } & {\color{equalParamColor} 1 } \\
{\color{equalParamColor} number of epochs } & {\color{equalParamColor} 500 } & {\color{equalParamColor} 500 } & {\color{equalParamColor} 500 } & {\color{equalParamColor} 500 } & {\color{equalParamColor} 500 } \\
batch size               & 1000    & 100     & 100     & 1000    & 1000    \\
optimizer                & Adam    & cAdam   & cAdam   & cAdam   & Adam    \\
learning rate            & 0.01    & 0.001   & 0.001   & 0.01    & 0.01    \\
{\color{equalParamColor} $\varepsilon$ } & {\color{equalParamColor} $10^{-7}$ } & {\color{equalParamColor} $10^{-7}$ } & {\color{equalParamColor} $10^{-7}$ } & {\color{equalParamColor} $10^{-7}$ } & {\color{equalParamColor} $10^{-7}$ } \\
\hline
\caption{best settings regarding \textit{final val loss mae avg} for the chemReg cAdam dataset}
\label{table:final_val_loss_mae_avg_best_chemreg_cadam}
\end{longtable}

% best parameter values regarding \texttt{final_val_loss_mae_avg}
\begin{longtable}{|l|c|c|c|c|c|c|c|c|c|c|c|c|c|c|c|c|c|c|c|}
\hline
\textbf{parameter name} & \multicolumn{6}{c|}{\textbf{parameter values}} & \multicolumn{6}{c|}{\textbf{win ratios in \%}} & \multicolumn{6}{c|}{\textbf{avg. differences}} & \textbf{best value} \\
\hline
neurons per layer & \multicolumn{2}{c:}{(40, 20)} & \multicolumn{2}{c:}{(50, 10)} & \multicolumn{2}{c|}{(30, 30, 10)} & \multicolumn{2}{c:}{29.2} & \multicolumn{2}{c:}{50.0} & \multicolumn{2}{c|}{20.8} & \multicolumn{2}{c:}{0.005} & \multicolumn{2}{c:}{0.003} & \multicolumn{2}{c|}{0.007} & (50, 10) \\
activation functions & \multicolumn{3}{c:}{ReLU} & \multicolumn{3}{c|}{sigmoid} & \multicolumn{3}{c:}{50.0} & \multicolumn{3}{c|}{50.0} & \multicolumn{3}{c:}{0.003} & \multicolumn{3}{c|}{0.006} & ReLU \\
loss function & \multicolumn{3}{c:}{MSE} & \multicolumn{3}{c|}{log cosh} & \multicolumn{3}{c:}{31.9} & \multicolumn{3}{c|}{68.1} & \multicolumn{3}{c:}{0.004} & \multicolumn{3}{c|}{0.002} & log cosh \\
batch size & \multicolumn{2}{c:}{100} & \multicolumn{2}{c:}{1000} & \multicolumn{2}{c|}{10000} & \multicolumn{2}{c:}{52.1} & \multicolumn{2}{c:}{45.8} & \multicolumn{2}{c|}{2.1} & \multicolumn{2}{c:}{0.005} & \multicolumn{2}{c:}{0.005} & \multicolumn{2}{c|}{0.032} & \textit{unclear} \\
optimizer & \multicolumn{3}{c:}{Adam} & \multicolumn{3}{c|}{cAdam} & \multicolumn{3}{c:}{41.7} & \multicolumn{3}{c|}{58.3} & \multicolumn{3}{c:}{0.005} & \multicolumn{3}{c|}{0.002} & cAdam \\
learning rate & \multicolumn{3}{c:}{0.01} & \multicolumn{3}{c|}{0.001} & \multicolumn{3}{c:}{63.9} & \multicolumn{3}{c|}{36.1} & \multicolumn{3}{c:}{0.004} & \multicolumn{3}{c|}{0.015} & 0.01 \\
\hline

\caption{parameter influence regarding \textit{final val loss mae avg} for the chemReg cAdam dataset}
\label{table:final_val_loss_mae_avg_ratios_chemreg_cadam}
\end{longtable}



\begin{longtable}{|l|l|l|l|l|>{\columncolor{worstColumnColor}}l|}
\hline
\textbf{parameter name} & \multicolumn{5}{c|}{\textbf{worst values}} \\
\hline
\textit{'final val loss mae avg'} (avg) & 0.19109 & 0.19588 & 0.20277 & 0.21157 & 0.22639 \\
test loss mae avg        & 0.18477 & 0.19365 & 0.19938 & 0.20834 & 0.22996 \\
test loss avg            & 0.034141 & 0.018634 & 0.019747 & 0.043405 & 0.052883 \\
training time avg        & 17.242  & 17.453  & 17.19   & 17.624  & 16.914  \\
neurons per layer        & (50, 10) & (50, 10) & (40, 20) & (30, 30, 10) & (40, 20) \\
{\color{equalParamColor} activation functions } & {\color{equalParamColor} sigmoid } & {\color{equalParamColor} sigmoid } & {\color{equalParamColor} sigmoid } & {\color{equalParamColor} sigmoid } & {\color{equalParamColor} sigmoid } \\
{\color{equalParamColor} last activation function } & {\color{equalParamColor} linear } & {\color{equalParamColor} linear } & {\color{equalParamColor} linear } & {\color{equalParamColor} linear } & {\color{equalParamColor} linear } \\
loss function            & MSE     & log cosh & log cosh & MSE     & MSE     \\
{\color{equalParamColor} training data percentage } & {\color{equalParamColor} 1 } & {\color{equalParamColor} 1 } & {\color{equalParamColor} 1 } & {\color{equalParamColor} 1 } & {\color{equalParamColor} 1 } \\
{\color{equalParamColor} number of epochs } & {\color{equalParamColor} 500 } & {\color{equalParamColor} 500 } & {\color{equalParamColor} 500 } & {\color{equalParamColor} 500 } & {\color{equalParamColor} 500 } \\
{\color{equalParamColor} batch size } & {\color{equalParamColor} 10000 } & {\color{equalParamColor} 10000 } & {\color{equalParamColor} 10000 } & {\color{equalParamColor} 10000 } & {\color{equalParamColor} 10000 } \\
{\color{equalParamColor} optimizer } & {\color{equalParamColor} Adam } & {\color{equalParamColor} Adam } & {\color{equalParamColor} Adam } & {\color{equalParamColor} Adam } & {\color{equalParamColor} Adam } \\
{\color{equalParamColor} learning rate } & {\color{equalParamColor} 0.001 } & {\color{equalParamColor} 0.001 } & {\color{equalParamColor} 0.001 } & {\color{equalParamColor} 0.001 } & {\color{equalParamColor} 0.001 } \\
{\color{equalParamColor} $\varepsilon$ } & {\color{equalParamColor} $10^{-7}$ } & {\color{equalParamColor} $10^{-7}$ } & {\color{equalParamColor} $10^{-7}$ } & {\color{equalParamColor} $10^{-7}$ } & {\color{equalParamColor} $10^{-7}$ } \\
\hline

\caption{worst settings regarding \textit{final val loss mae avg} for the chemReg cAdam dataset}
\label{table:final_val_loss_mae_avg_worst_chemreg_cadam}
\end{longtable}


%%%%%%%%%%%%%%%%%%%%%%%%%%%%%%%%%%%%%%%%%%%%%%%%%%%%%%%%%%%%%%%%%%%%%%%
\section{MNIST Bachelorthesis new result tables}
\subsection{accuracy}
\begin{longtable}{|l|>{\columncolor{bestColumnColor}}l|l|l|l|l|}
\hline
\textbf{parameter name} & \multicolumn{5}{c|}{\textbf{best values}} \\
\hline
\textit{final val accuracy avg} &    0.97 &  0.9691 & 0.96905 & 0.96885 &  0.9688 \\
test accuracy avg        & 0.97006 & 0.96906 & 0.9681  & 0.97048 & 0.97084 \\
training time avg        & 46.666  & 43.602  & 86.105  & 86.874  & 23.122  \\
{\color{equalParamColor} neurons per layer } & {\color{equalParamColor} (50, 10) } & {\color{equalParamColor} (50, 10) } & {\color{equalParamColor} (50, 10) } & {\color{equalParamColor} (50, 10) } & {\color{equalParamColor} (50, 10) } \\
{\color{equalParamColor} activation functions } & {\color{equalParamColor} ReLU } & {\color{equalParamColor} ReLU } & {\color{equalParamColor} ReLU } & {\color{equalParamColor} ReLU } & {\color{equalParamColor} ReLU } \\
last activation function & sigmoid & sigmoid & sigmoid & softmax & softmax \\
{\color{equalParamColor} loss function } & {\color{equalParamColor} cat-cross } & {\color{equalParamColor} cat-cross } & {\color{equalParamColor} cat-cross } & {\color{equalParamColor} cat-cross } & {\color{equalParamColor} cat-cross } \\
{\color{equalParamColor} training data percentage } & {\color{equalParamColor} 1 } & {\color{equalParamColor} 1 } & {\color{equalParamColor} 1 } & {\color{equalParamColor} 1 } & {\color{equalParamColor} 1 } \\
number of epochs         & 50      & 25      & 50      & 50      & 25      \\
{\color{equalParamColor} batch size } & {\color{equalParamColor} 100 } & {\color{equalParamColor} 100 } & {\color{equalParamColor} 100 } & {\color{equalParamColor} 100 } & {\color{equalParamColor} 100 } \\
optimizer                & Adam    & cAdam   & cAdam   & cAdam   & Adam    \\
learning rate            & 0.001   & 0.1     & 0.1     & 0.1     & 0.1     \\
$\varepsilon$            & $10^{-7}$ & 1.0     & 1.0     & 1.0     & 1.0     \\
\hline

\caption{best settings regarding \textit{final val accuracy avg} for the MNIST BSc dataset}
\label{table:final_val_accuracy_avg_best_mnist_bsc}
\end{longtable}


\begin{longtable}{|l|l|l|l|l|>{\columncolor{worstColumnColor}}l|}
\hline
\textbf{parameter name} & \multicolumn{5}{c|}{\textbf{worst values}} \\
\hline
\textit{final val accuracy avg} & 0.085333 &  0.0852 & 0.085183 & 0.084467 &  0.0833 \\
test accuracy avg        & 0.0845  & 0.0821  & 0.08686 & 0.08748 & 0.08516 \\
training time avg        & 1.4319  & 6.3722  & 3.1055  & 3.1968  & 1.3563  \\
neurons per layer        & (32,)   & (20, 15, 10) & (32,)   & (32,)   & (50, 10) \\
activation functions     & ReLU    & ReLU    & ReLU    & sigmoid & ReLU    \\
last activation function & softmax & sigmoid & sigmoid & sigmoid & sigmoid \\
loss function            & MSE     & MSE     & MSE     & cat-cross & MSE     \\
{\color{equalParamColor} training data percentage } & {\color{equalParamColor} 1 } & {\color{equalParamColor} 1 } & {\color{equalParamColor} 1 } & {\color{equalParamColor} 1 } & {\color{equalParamColor} 1 } \\
number of epochs         & 5       & 5       & 25      & 25      & 5       \\
batch size               & 1000    & 100     & 10000   & 10000   & 10000   \\
optimizer                & Adam    & Adam    & Adam    & Adam    & cAdam   \\
{\color{equalParamColor} learning rate } & {\color{equalParamColor} 0.001 } & {\color{equalParamColor} 0.001 } & {\color{equalParamColor} 0.001 } & {\color{equalParamColor} 0.001 } & {\color{equalParamColor} 0.001 } \\
{\color{equalParamColor} $\varepsilon$ } & {\color{equalParamColor} 1.0 } & {\color{equalParamColor} 1.0 } & {\color{equalParamColor} 1.0 } & {\color{equalParamColor} 1.0 } & {\color{equalParamColor} 1.0 } \\
\hline

\caption{worst settings regarding \textit{final val accuracy avg} for the MNIST BSc dataset}
\label{table:final_val_accuracy_avg_worst_mnist_bsc}
\end{longtable}


% best parameter values regarding \texttt{final_val_accuracy_avg}
\begin{longtable}{|l|c|c|c|c|c|c|c|c|c|c|c|c|c|c|c|c|c|c|c|}
\hline
\textbf{parameter name} & \multicolumn{6}{c|}{\textbf{parameter values}} & \multicolumn{6}{c|}{\textbf{win ratios in \%}} & \multicolumn{6}{c|}{\textbf{avg. differences}} & \textbf{best value} \\
\hline
neurons per layer & \multicolumn{2}{c:}{(32,)} & \multicolumn{2}{c:}{(50, 10)} & \multicolumn{2}{c|}{(20, 15, 10)} & \multicolumn{2}{c:}{63.8} & \multicolumn{2}{c:}{31.4} & \multicolumn{2}{c|}{4.9} & \multicolumn{2}{c:}{0.003} & \multicolumn{2}{c:}{0.117} & \multicolumn{2}{c|}{0.17} & (32,) \\
activation functions & \multicolumn{3}{c:}{ReLU} & \multicolumn{3}{c|}{sigmoid} & \multicolumn{3}{c:}{73.1} & \multicolumn{3}{c|}{26.9} & \multicolumn{3}{c:}{0.062} & \multicolumn{3}{c|}{0.114} & ReLU \\
last activation function & \multicolumn{3}{c:}{softmax} & \multicolumn{3}{c|}{sigmoid} & \multicolumn{3}{c:}{62.3} & \multicolumn{3}{c|}{37.7} & \multicolumn{3}{c:}{0.007} & \multicolumn{3}{c|}{0.069} & softmax \\
loss function & \multicolumn{3}{c:}{cat-cross} & \multicolumn{3}{c|}{MSE} & \multicolumn{3}{c:}{86.7} & \multicolumn{3}{c|}{13.3} & \multicolumn{3}{c:}{0.001} & \multicolumn{3}{c|}{0.221} & cat-cross \\
number of epochs & \multicolumn{2}{c:}{5} & \multicolumn{2}{c:}{25} & \multicolumn{2}{c|}{50} & \multicolumn{2}{c:}{6.8} & \multicolumn{2}{c:}{19.2} & \multicolumn{2}{c|}{74.0} & \multicolumn{2}{c:}{0.159} & \multicolumn{2}{c:}{0.047} & \multicolumn{2}{c|}{0.006} & 50 \\
batch size & \multicolumn{2}{c:}{100} & \multicolumn{2}{c:}{1000} & \multicolumn{2}{c|}{10000} & \multicolumn{2}{c:}{74.5} & \multicolumn{2}{c:}{19.4} & \multicolumn{2}{c|}{6.0} & \multicolumn{2}{c:}{0.023} & \multicolumn{2}{c:}{0.131} & \multicolumn{2}{c|}{0.283} & 100 \\
optimizer & \multicolumn{3}{c:}{Adam} & \multicolumn{3}{c|}{cAdam} & \multicolumn{3}{c:}{34.6} & \multicolumn{3}{c|}{65.4} & \multicolumn{3}{c:}{0.031} & \multicolumn{3}{c|}{0.01} & cAdam \\
learning rate & \multicolumn{2}{c:}{0.1} & \multicolumn{2}{c:}{0.01} & \multicolumn{2}{c|}{0.001} & \multicolumn{2}{c:}{46.9} & \multicolumn{2}{c:}{35.1} & \multicolumn{2}{c|}{18.1} & \multicolumn{2}{c:}{0.175} & \multicolumn{2}{c:}{0.096} & \multicolumn{2}{c|}{0.202} & \textit{unclear} \\
$\varepsilon$ & \multicolumn{3}{c:}{1.0} & \multicolumn{3}{c|}{$10^{-7}$} & \multicolumn{3}{c:}{16.2} & \multicolumn{3}{c|}{83.8} & \multicolumn{3}{c:}{0.437} & \multicolumn{3}{c|}{0.058} & $10^{-7}$ \\
\hline

\caption{parameter influence regarding \textit{final val accuracy avg} for the MNIST BSc dataset}
\label{table:final_val_accuracy_avg_ratios_mnist_bsc}
\end{longtable}



%%%%%%%%%%%%%%%%%%%%%%%%%%%%%%%%%%%%%%%%%%%%%%%%%%%%%%%%%%%%%%%%%%%%%%%
\section{MNIST Adam variant comparison}
\subsection{revised, longer experiments, 1 run}
\subsection{training time}
\begin{longtable}{|l|>{\columncolor{bestColumnColor}}l|l|l|l|l|}
\hline
\textbf{parameter name} & \multicolumn{5}{c|}{\textbf{best values}} \\
\hline
\textit{training time avg} &  55.489 &  62.251 &  92.568 &   96.35 \\
{\color{equalParamColor} neurons per layer } & {\color{equalParamColor} (50, 10) } & {\color{equalParamColor} (50, 10) } & {\color{equalParamColor} (50, 10) } & {\color{equalParamColor} (50, 10) } \\
{\color{equalParamColor} activation functions } & {\color{equalParamColor} ReLU } & {\color{equalParamColor} ReLU } & {\color{equalParamColor} ReLU } & {\color{equalParamColor} ReLU } \\
last activation function & sigmoid & softmax & sigmoid & softmax \\
{\color{equalParamColor} loss function } & {\color{equalParamColor} cat-cross } & {\color{equalParamColor} cat-cross } & {\color{equalParamColor} cat-cross } & {\color{equalParamColor} cat-cross } \\
{\color{equalParamColor} training data percentage } & {\color{equalParamColor} 1.0 } & {\color{equalParamColor} 1.0 } & {\color{equalParamColor} 1.0 } & {\color{equalParamColor} 1.0 } \\
{\color{equalParamColor} number of epochs } & {\color{equalParamColor} 250 } & {\color{equalParamColor} 250 } & {\color{equalParamColor} 250 } & {\color{equalParamColor} 250 } \\
{\color{equalParamColor} batch size } & {\color{equalParamColor} 100 } & {\color{equalParamColor} 100 } & {\color{equalParamColor} 100 } & {\color{equalParamColor} 100 } \\
optimizer                & Adam    & Adam    & my Adam & my Adam \\
{\color{equalParamColor} learning rate } & {\color{equalParamColor} 0.001 } & {\color{equalParamColor} 0.001 } & {\color{equalParamColor} 0.001 } & {\color{equalParamColor} 0.001 } \\
{\color{equalParamColor} $\varepsilon$ } & {\color{equalParamColor} $10^{-7}$ } & {\color{equalParamColor} $10^{-7}$ } & {\color{equalParamColor} $10^{-7}$ } & {\color{equalParamColor} $10^{-7}$ } \\
\hline

\caption{best settings regarding \textit{training time avg} for the cadam variants dataset}
\label{table:variant_training_time_avg_best_cadam_variants}
\end{longtable}


\begin{longtable}{|l|l|l|l|l|>{\columncolor{worstColumnColor}}l|}
\hline
\textbf{parameter name} & \multicolumn{5}{c|}{\textbf{worst values}} \\
\hline
\textit{training time avg} &   96.35 &  104.57 &  110.17 &  113.87 &  115.73 \\
{\color{equalParamColor} neurons per layer } & {\color{equalParamColor} (50, 10) } & {\color{equalParamColor} (50, 10) } & {\color{equalParamColor} (50, 10) } & {\color{equalParamColor} (50, 10) } & {\color{equalParamColor} (50, 10) } \\
{\color{equalParamColor} activation functions } & {\color{equalParamColor} ReLU } & {\color{equalParamColor} ReLU } & {\color{equalParamColor} ReLU } & {\color{equalParamColor} ReLU } & {\color{equalParamColor} ReLU } \\
last activation function & softmax & softmax & sigmoid & softmax & sigmoid \\
{\color{equalParamColor} loss function } & {\color{equalParamColor} cat-cross } & {\color{equalParamColor} cat-cross } & {\color{equalParamColor} cat-cross } & {\color{equalParamColor} cat-cross } & {\color{equalParamColor} cat-cross } \\
{\color{equalParamColor} training data percentage } & {\color{equalParamColor} 1.0 } & {\color{equalParamColor} 1.0 } & {\color{equalParamColor} 1.0 } & {\color{equalParamColor} 1.0 } & {\color{equalParamColor} 1.0 } \\
{\color{equalParamColor} number of epochs } & {\color{equalParamColor} 250 } & {\color{equalParamColor} 250 } & {\color{equalParamColor} 250 } & {\color{equalParamColor} 250 } & {\color{equalParamColor} 250 } \\
{\color{equalParamColor} batch size } & {\color{equalParamColor} 100 } & {\color{equalParamColor} 100 } & {\color{equalParamColor} 100 } & {\color{equalParamColor} 100 } & {\color{equalParamColor} 100 } \\
optimizer                & my Adam & c adam hat & c adam hat & cAdam   & cAdam   \\
{\color{equalParamColor} learning rate } & {\color{equalParamColor} 0.001 } & {\color{equalParamColor} 0.001 } & {\color{equalParamColor} 0.001 } & {\color{equalParamColor} 0.001 } & {\color{equalParamColor} 0.001 } \\
{\color{equalParamColor} $\varepsilon$ } & {\color{equalParamColor} $10^{-7}$ } & {\color{equalParamColor} $10^{-7}$ } & {\color{equalParamColor} $10^{-7}$ } & {\color{equalParamColor} $10^{-7}$ } & {\color{equalParamColor} $10^{-7}$ } \\
\hline

\caption{worst settings regarding \textit{training time avg} for the cadam variants dataset}
\label{table:variant_training_time_avg_worst_cadam_variants}
\end{longtable}


% best parameter values regarding \texttt{training_time_avg}
\begin{longtable}{|l|c|c|c|c|c|c|c|c|c|c|c|c|c|c|c|c|c|c|c|}
\hline
\textbf{parameter name} & \multicolumn{6}{c|}{\textbf{parameter values}} & \multicolumn{6}{c|}{\textbf{win ratios in \%}} & \multicolumn{6}{c|}{\textbf{avg. differences in \%}} & \textbf{best value} \\
\hline
last activation function & \multicolumn{3}{c:}{softmax} & \multicolumn{3}{c|}{sigmoid} & \multicolumn{3}{c:}{50.0} & \multicolumn{3}{c|}{50.0} & \multicolumn{3}{c:}{4.068} & \multicolumn{3}{c|}{1.747} & sigmoid \\
optimizer & \multicolumn{1}{c:}{Adam} & \multicolumn{1}{c:}{my Adam} & \multicolumn{1}{c:}{cAdam} & \multicolumn{1}{c|}{cAdam_hat} & \multicolumn{1}{c:}{100.0} & \multicolumn{1}{c:}{  0} & \multicolumn{1}{c:}{  0} & \multicolumn{1}{c|}{  0} & \multicolumn{1}{c:}{    0} & \multicolumn{1}{c:}{60.799} & \multicolumn{1}{c:}{95.747} & \multicolumn{1}{c|}{83.259} & Adam \\
\hline

\caption{parameter influence regarding \textit{training time avg} for the cadam variants dataset}
\label{table:variant_training_time_avg_ratios_cadam_variants}
\end{longtable}


\subsection{accuracy}
\begin{longtable}{|l|>{\columncolor{bestColumnColor}}l|l|l|l|l|}
\hline
\textbf{parameter name} & \multicolumn{5}{c|}{\textbf{best values}} \\
\hline
\textit{test accuracy avg} &  0.1943 &  0.1941 & 0.19394 &  0.1939 &  0.1938 \\
final val accuracy avg   & 0.19403 & 0.19362 & 0.19413 & 0.19363 & 0.1939  \\
final val accuracy std   & 0.38807 & 0.38723 & 0.38827 & 0.38727 & 0.3878  \\
{\color{equalParamColor} final val accuracy min } & {\color{equalParamColor} 0.0 } & {\color{equalParamColor} 0.0 } & {\color{equalParamColor} 0.0 } & {\color{equalParamColor} 0.0 } & {\color{equalParamColor} 0.0 } \\
final val accuracy max   & 0.97017 & 0.96808 & 0.97067 & 0.96817 & 0.9695  \\
training time avg        & 96.35   & 113.87  & 92.568  & 55.489  & 115.73  \\
{\color{equalParamColor} neurons per layer } & {\color{equalParamColor} (50, 10) } & {\color{equalParamColor} (50, 10) } & {\color{equalParamColor} (50, 10) } & {\color{equalParamColor} (50, 10) } & {\color{equalParamColor} (50, 10) } \\
{\color{equalParamColor} activation functions } & {\color{equalParamColor} ReLU } & {\color{equalParamColor} ReLU } & {\color{equalParamColor} ReLU } & {\color{equalParamColor} ReLU } & {\color{equalParamColor} ReLU } \\
last activation function & softmax & softmax & sigmoid & sigmoid & sigmoid \\
{\color{equalParamColor} loss function } & {\color{equalParamColor} cat-cross } & {\color{equalParamColor} cat-cross } & {\color{equalParamColor} cat-cross } & {\color{equalParamColor} cat-cross } & {\color{equalParamColor} cat-cross } \\
{\color{equalParamColor} training data percentage } & {\color{equalParamColor} 1.0 } & {\color{equalParamColor} 1.0 } & {\color{equalParamColor} 1.0 } & {\color{equalParamColor} 1.0 } & {\color{equalParamColor} 1.0 } \\
{\color{equalParamColor} number of epochs } & {\color{equalParamColor} 250 } & {\color{equalParamColor} 250 } & {\color{equalParamColor} 250 } & {\color{equalParamColor} 250 } & {\color{equalParamColor} 250 } \\
{\color{equalParamColor} batch size } & {\color{equalParamColor} 100 } & {\color{equalParamColor} 100 } & {\color{equalParamColor} 100 } & {\color{equalParamColor} 100 } & {\color{equalParamColor} 100 } \\
optimizer                & my Adam & cAdam   & my Adam & Adam    & cAdam   \\
{\color{equalParamColor} learning rate } & {\color{equalParamColor} 0.001 } & {\color{equalParamColor} 0.001 } & {\color{equalParamColor} 0.001 } & {\color{equalParamColor} 0.001 } & {\color{equalParamColor} 0.001 } \\
{\color{equalParamColor} $\varepsilon$ } & {\color{equalParamColor} $10^{-7}$ } & {\color{equalParamColor} $10^{-7}$ } & {\color{equalParamColor} $10^{-7}$ } & {\color{equalParamColor} $10^{-7}$ } & {\color{equalParamColor} $10^{-7}$ } \\
\hline

\caption{best settings regarding \textit{test accuracy avg} for the cadam variants dataset}
\label{table:variant_test_accuracy_avg_best_cadam_variants}
\end{longtable}


\begin{longtable}{|l|l|l|l|l|>{\columncolor{worstColumnColor}}l|}
\hline
\textbf{parameter name} & \multicolumn{5}{c|}{\textbf{worst values}} \\
\hline
\textit{test accuracy avg} &  0.1939 &  0.1938 & 0.19372 & 0.13374 & 0.13306 \\
final val accuracy avg   & 0.19363 & 0.1939  & 0.1939  & 0.13457 & 0.13483 \\
final val accuracy std   & 0.38727 & 0.3878  & 0.3878  & 0.26913 & 0.26967 \\
{\color{equalParamColor} final val accuracy min } & {\color{equalParamColor} 0.0 } & {\color{equalParamColor} 0.0 } & {\color{equalParamColor} 0.0 } & {\color{equalParamColor} 0.0 } & {\color{equalParamColor} 0.0 } \\
final val accuracy max   & 0.96817 & 0.9695  & 0.9695  & 0.67283 & 0.67417 \\
training time avg        & 55.489  & 115.73  & 62.251  & 110.17  & 104.57  \\
{\color{equalParamColor} neurons per layer } & {\color{equalParamColor} (50, 10) } & {\color{equalParamColor} (50, 10) } & {\color{equalParamColor} (50, 10) } & {\color{equalParamColor} (50, 10) } & {\color{equalParamColor} (50, 10) } \\
{\color{equalParamColor} activation functions } & {\color{equalParamColor} ReLU } & {\color{equalParamColor} ReLU } & {\color{equalParamColor} ReLU } & {\color{equalParamColor} ReLU } & {\color{equalParamColor} ReLU } \\
last activation function & sigmoid & sigmoid & softmax & sigmoid & softmax \\
{\color{equalParamColor} loss function } & {\color{equalParamColor} cat-cross } & {\color{equalParamColor} cat-cross } & {\color{equalParamColor} cat-cross } & {\color{equalParamColor} cat-cross } & {\color{equalParamColor} cat-cross } \\
{\color{equalParamColor} training data percentage } & {\color{equalParamColor} 1.0 } & {\color{equalParamColor} 1.0 } & {\color{equalParamColor} 1.0 } & {\color{equalParamColor} 1.0 } & {\color{equalParamColor} 1.0 } \\
{\color{equalParamColor} number of epochs } & {\color{equalParamColor} 250 } & {\color{equalParamColor} 250 } & {\color{equalParamColor} 250 } & {\color{equalParamColor} 250 } & {\color{equalParamColor} 250 } \\
{\color{equalParamColor} batch size } & {\color{equalParamColor} 100 } & {\color{equalParamColor} 100 } & {\color{equalParamColor} 100 } & {\color{equalParamColor} 100 } & {\color{equalParamColor} 100 } \\
optimizer                & Adam    & cAdam   & Adam    & c adam hat & c adam hat \\
{\color{equalParamColor} learning rate } & {\color{equalParamColor} 0.001 } & {\color{equalParamColor} 0.001 } & {\color{equalParamColor} 0.001 } & {\color{equalParamColor} 0.001 } & {\color{equalParamColor} 0.001 } \\
{\color{equalParamColor} $\varepsilon$ } & {\color{equalParamColor} $10^{-7}$ } & {\color{equalParamColor} $10^{-7}$ } & {\color{equalParamColor} $10^{-7}$ } & {\color{equalParamColor} $10^{-7}$ } & {\color{equalParamColor} $10^{-7}$ } \\
\hline

\caption{worst settings regarding \textit{test accuracy avg} for the cadam variants dataset}
\label{table:variant_test_accuracy_avg_worst_cadam_variants}
\end{longtable}


% best parameter values regarding \texttt{test_accuracy_avg}
\begin{longtable}{|l|c|c|c|c|c|c|c|c|c|c|c|c|c|c|c|c|c|c|c|}
\hline
\textbf{parameter name} & \multicolumn{6}{c|}{\textbf{parameter values}} & \multicolumn{6}{c|}{\textbf{win ratios in \%}} & \multicolumn{6}{c|}{\textbf{avg. differences in \%}} & \textbf{best value} \\
\hline
last activation function & \multicolumn{3}{c:}{softmax} & \multicolumn{3}{c|}{sigmoid} & \multicolumn{3}{c:}{50.0} & \multicolumn{3}{c|}{50.0} & \multicolumn{3}{c:}{0.15} & \multicolumn{3}{c|}{0.085} & sigmoid \\
optimizer & \multicolumn{1}{c:}{Adam} & \multicolumn{1}{c:}{my Adam} & \multicolumn{1}{c:}{cAdam} & \multicolumn{1}{c|}{cAdam_hat} & \multicolumn{1}{c:}{  0} & \multicolumn{1}{c:}{100.0} & \multicolumn{1}{c:}{  0} & \multicolumn{1}{c|}{  0} & \multicolumn{1}{c:}{0.16} & \multicolumn{1}{c:}{    0} & \multicolumn{1}{c:}{0.088} & \multicolumn{1}{c|}{31.279} & my Adam \\
\hline

\caption{parameter influence regarding \textit{test accuracy avg} for the cadam variants dataset}
\label{table:variant_test_accuracy_avg_ratios_cadam_variants}
\end{longtable}




%%%%%%%%%%%%%%%%%%%%%%%%%%%%%%%%%%%%%%%%%%%%%%%%%%%%%%%%%%%%%%%%%%%%%%%
\section{MNIST with \% differences}
\subsection{training time}
\begin{longtable}{|l|>{\columncolor{bestColumnColor}}l|l|l|l|l|}
\hline
\textbf{parameter name} & \multicolumn{5}{c|}{\textbf{best values}} \\
\hline
\textit{training time avg} &  1.0347 &  1.0368 &  1.0398 &  1.0412 &  1.0446 \\
{\color{equalParamColor} neurons per layer } & {\color{equalParamColor} (32,) } & {\color{equalParamColor} (32,) } & {\color{equalParamColor} (32,) } & {\color{equalParamColor} (32,) } & {\color{equalParamColor} (32,) } \\
activation functions     & sigmoid & sigmoid & ReLU    & ReLU    & sigmoid \\
{\color{equalParamColor} last activation function } & {\color{equalParamColor} sigmoid } & {\color{equalParamColor} sigmoid } & {\color{equalParamColor} sigmoid } & {\color{equalParamColor} sigmoid } & {\color{equalParamColor} sigmoid } \\
{\color{equalParamColor} loss function } & {\color{equalParamColor} MSE } & {\color{equalParamColor} MSE } & {\color{equalParamColor} MSE } & {\color{equalParamColor} MSE } & {\color{equalParamColor} MSE } \\
{\color{equalParamColor} training data percentage } & {\color{equalParamColor} 1 } & {\color{equalParamColor} 1 } & {\color{equalParamColor} 1 } & {\color{equalParamColor} 1 } & {\color{equalParamColor} 1 } \\
{\color{equalParamColor} number of epochs } & {\color{equalParamColor} 5 } & {\color{equalParamColor} 5 } & {\color{equalParamColor} 5 } & {\color{equalParamColor} 5 } & {\color{equalParamColor} 5 } \\
{\color{equalParamColor} batch size } & {\color{equalParamColor} 10000 } & {\color{equalParamColor} 10000 } & {\color{equalParamColor} 10000 } & {\color{equalParamColor} 10000 } & {\color{equalParamColor} 10000 } \\
{\color{equalParamColor} optimizer } & {\color{equalParamColor} Adam } & {\color{equalParamColor} Adam } & {\color{equalParamColor} Adam } & {\color{equalParamColor} Adam } & {\color{equalParamColor} Adam } \\
learning rate            & 0.001   & 0.01    & 0.001   & 0.1     & 0.1     \\
$\varepsilon$            & $10^{-7}$ & $10^{-7}$ & $10^{-7}$ & 1.0     & 1.0     \\
\hline

\caption{best settings regarding \textit{training time avg} for the MNIST revised dataset}
\label{table:training_time_avg_best_mnist_revised}
\end{longtable}


\begin{longtable}{|l|l|l|l|l|>{\columncolor{worstColumnColor}}l|}
\hline
\textbf{parameter name} & \multicolumn{5}{c|}{\textbf{worst values}} \\
\hline
\textit{training time avg} &  106.16 &  106.61 &  106.64 &  106.92 &  106.93 \\
{\color{equalParamColor} neurons per layer } & {\color{equalParamColor} (20, 15, 10) } & {\color{equalParamColor} (20, 15, 10) } & {\color{equalParamColor} (20, 15, 10) } & {\color{equalParamColor} (20, 15, 10) } & {\color{equalParamColor} (20, 15, 10) } \\
{\color{equalParamColor} activation functions } & {\color{equalParamColor} sigmoid } & {\color{equalParamColor} sigmoid } & {\color{equalParamColor} sigmoid } & {\color{equalParamColor} sigmoid } & {\color{equalParamColor} sigmoid } \\
{\color{equalParamColor} last activation function } & {\color{equalParamColor} softmax } & {\color{equalParamColor} softmax } & {\color{equalParamColor} softmax } & {\color{equalParamColor} softmax } & {\color{equalParamColor} softmax } \\
{\color{equalParamColor} loss function } & {\color{equalParamColor} cat-cross } & {\color{equalParamColor} cat-cross } & {\color{equalParamColor} cat-cross } & {\color{equalParamColor} cat-cross } & {\color{equalParamColor} cat-cross } \\
{\color{equalParamColor} training data percentage } & {\color{equalParamColor} 1 } & {\color{equalParamColor} 1 } & {\color{equalParamColor} 1 } & {\color{equalParamColor} 1 } & {\color{equalParamColor} 1 } \\
{\color{equalParamColor} number of epochs } & {\color{equalParamColor} 50 } & {\color{equalParamColor} 50 } & {\color{equalParamColor} 50 } & {\color{equalParamColor} 50 } & {\color{equalParamColor} 50 } \\
{\color{equalParamColor} batch size } & {\color{equalParamColor} 100 } & {\color{equalParamColor} 100 } & {\color{equalParamColor} 100 } & {\color{equalParamColor} 100 } & {\color{equalParamColor} 100 } \\
{\color{equalParamColor} optimizer } & {\color{equalParamColor} cAdam } & {\color{equalParamColor} cAdam } & {\color{equalParamColor} cAdam } & {\color{equalParamColor} cAdam } & {\color{equalParamColor} cAdam } \\
learning rate            & 0.001   & 0.01    & 0.1     & 0.01    & 0.1     \\
$\varepsilon$            & 1.0     & 1.0     & 1.0     & $10^{-7}$ & $10^{-7}$ \\
\hline

\caption{worst settings regarding \textit{training time avg} for the MNIST revised dataset}
\label{table:training_time_avg_worst_mnist_revised}
\end{longtable}


% best parameter values regarding \texttt{training_time_avg}
\begin{longtable}{|l|c|c|c|c|c|c|c|c|c|c|c|c|c|c|c|c|c|c|c|}
\hline
\textbf{parameter name} & \multicolumn{6}{c|}{\textbf{parameter values}} & \multicolumn{6}{c|}{\textbf{win ratios in \%}} & \multicolumn{6}{c|}{\textbf{avg. differences in \%}} & \textbf{best value} \\
\hline
neurons per layer & \multicolumn{2}{c:}{(32,)} & \multicolumn{2}{c:}{(50, 10)} & \multicolumn{2}{c|}{(20, 15, 10)} & \multicolumn{2}{c:}{98.6} & \multicolumn{2}{c:}{1.2} & \multicolumn{2}{c|}{0.2} & \multicolumn{2}{c:}{0.036} & \multicolumn{2}{c:}{12.017} & \multicolumn{2}{c|}{22.725} & (32,) \\
activation functions & \multicolumn{3}{c:}{ReLU} & \multicolumn{3}{c|}{sigmoid} & \multicolumn{3}{c:}{49.9} & \multicolumn{3}{c|}{50.1} & \multicolumn{3}{c:}{0.716} & \multicolumn{3}{c|}{0.602} & sigmoid \\
last activation function & \multicolumn{3}{c:}{softmax} & \multicolumn{3}{c|}{sigmoid} & \multicolumn{3}{c:}{24.8} & \multicolumn{3}{c|}{75.2} & \multicolumn{3}{c:}{1.296} & \multicolumn{3}{c|}{0.506} & sigmoid \\
loss function & \multicolumn{3}{c:}{cat-cross} & \multicolumn{3}{c|}{MSE} & \multicolumn{3}{c:}{52.6} & \multicolumn{3}{c|}{47.4} & \multicolumn{3}{c:}{1.241} & \multicolumn{3}{c|}{4.097} & cat-cross \\
number of epochs & \multicolumn{2}{c:}{5} & \multicolumn{2}{c:}{25} & \multicolumn{2}{c|}{50} & \multicolumn{2}{c:}{100.0} & \multicolumn{2}{c:}{  0} & \multicolumn{2}{c|}{  0} & \multicolumn{2}{c:}{    0} & \multicolumn{2}{c:}{264.684} & \multicolumn{2}{c|}{596.88} & 5 \\
batch size & \multicolumn{2}{c:}{100} & \multicolumn{2}{c:}{1000} & \multicolumn{2}{c|}{10000} & \multicolumn{2}{c:}{  0} & \multicolumn{2}{c:}{  0} & \multicolumn{2}{c|}{100.0} & \multicolumn{2}{c:}{779.878} & \multicolumn{2}{c:}{64.515} & \multicolumn{2}{c|}{    0} & 10000 \\
optimizer & \multicolumn{3}{c:}{Adam} & \multicolumn{3}{c|}{cAdam} & \multicolumn{3}{c:}{100.0} & \multicolumn{3}{c|}{  0} & \multicolumn{3}{c:}{    0} & \multicolumn{3}{c|}{39.945} & Adam \\
learning rate & \multicolumn{2}{c:}{0.1} & \multicolumn{2}{c:}{0.01} & \multicolumn{2}{c|}{0.001} & \multicolumn{2}{c:}{34.7} & \multicolumn{2}{c:}{30.9} & \multicolumn{2}{c|}{34.4} & \multicolumn{2}{c:}{0.914} & \multicolumn{2}{c:}{1.126} & \multicolumn{2}{c|}{1.094} & 0.1 \\
$\varepsilon$ & \multicolumn{3}{c:}{1.0} & \multicolumn{3}{c|}{$10^{-7}$} & \multicolumn{3}{c:}{54.9} & \multicolumn{3}{c|}{45.1} & \multicolumn{3}{c:}{0.666} & \multicolumn{3}{c|}{0.686} & 1.0 \\
\hline

\caption{parameter influence regarding \textit{training time avg} for the MNIST revised dataset}
\label{table:training_time_avg_ratios_mnist_revised}
\end{longtable}



\subsection{accuracy}
\begin{longtable}{|l|>{\columncolor{bestColumnColor}}l|l|l|l|l|}
\hline
\textbf{parameter name} & \multicolumn{5}{c|}{\textbf{best values}} \\
\hline
\textit{final val accuracy avg} &    0.97 &  0.9691 & 0.96905 & 0.96885 &  0.9688 \\
test accuracy avg        & 0.97006 & 0.96906 & 0.9681  & 0.97048 & 0.97084 \\
training time avg        & 46.666  & 43.602  & 86.105  & 86.874  & 23.122  \\
{\color{equalParamColor} neurons per layer } & {\color{equalParamColor} (50, 10) } & {\color{equalParamColor} (50, 10) } & {\color{equalParamColor} (50, 10) } & {\color{equalParamColor} (50, 10) } & {\color{equalParamColor} (50, 10) } \\
{\color{equalParamColor} activation functions } & {\color{equalParamColor} ReLU } & {\color{equalParamColor} ReLU } & {\color{equalParamColor} ReLU } & {\color{equalParamColor} ReLU } & {\color{equalParamColor} ReLU } \\
last activation function & sigmoid & sigmoid & sigmoid & softmax & softmax \\
{\color{equalParamColor} loss function } & {\color{equalParamColor} cat-cross } & {\color{equalParamColor} cat-cross } & {\color{equalParamColor} cat-cross } & {\color{equalParamColor} cat-cross } & {\color{equalParamColor} cat-cross } \\
{\color{equalParamColor} training data percentage } & {\color{equalParamColor} 1 } & {\color{equalParamColor} 1 } & {\color{equalParamColor} 1 } & {\color{equalParamColor} 1 } & {\color{equalParamColor} 1 } \\
number of epochs         & 50      & 25      & 50      & 50      & 25      \\
{\color{equalParamColor} batch size } & {\color{equalParamColor} 100 } & {\color{equalParamColor} 100 } & {\color{equalParamColor} 100 } & {\color{equalParamColor} 100 } & {\color{equalParamColor} 100 } \\
optimizer                & Adam    & cAdam   & cAdam   & cAdam   & Adam    \\
learning rate            & 0.001   & 0.1     & 0.1     & 0.1     & 0.1     \\
$\varepsilon$            & $10^{-7}$ & 1.0     & 1.0     & 1.0     & 1.0     \\
\hline

\caption{best settings regarding \textit{final val accuracy avg} for the MNIST revised dataset}
\label{table:final_val_accuracy_avg_best_mnist_revised}
\end{longtable}


\begin{longtable}{|l|l|l|l|l|>{\columncolor{worstColumnColor}}l|}
\hline
\textbf{parameter name} & \multicolumn{5}{c|}{\textbf{worst values}} \\
\hline
\textit{final val accuracy avg} & 0.085333 &  0.0852 & 0.085183 & 0.084467 &  0.0833 \\
test accuracy avg        & 0.0845  & 0.0821  & 0.08686 & 0.08748 & 0.08516 \\
training time avg        & 1.4319  & 6.3722  & 3.1055  & 3.1968  & 1.3563  \\
neurons per layer        & (32,)   & (20, 15, 10) & (32,)   & (32,)   & (50, 10) \\
activation functions     & ReLU    & ReLU    & ReLU    & sigmoid & ReLU    \\
last activation function & softmax & sigmoid & sigmoid & sigmoid & sigmoid \\
loss function            & MSE     & MSE     & MSE     & cat-cross & MSE     \\
{\color{equalParamColor} training data percentage } & {\color{equalParamColor} 1 } & {\color{equalParamColor} 1 } & {\color{equalParamColor} 1 } & {\color{equalParamColor} 1 } & {\color{equalParamColor} 1 } \\
number of epochs         & 5       & 5       & 25      & 25      & 5       \\
batch size               & 1000    & 100     & 10000   & 10000   & 10000   \\
optimizer                & Adam    & Adam    & Adam    & Adam    & cAdam   \\
{\color{equalParamColor} learning rate } & {\color{equalParamColor} 0.001 } & {\color{equalParamColor} 0.001 } & {\color{equalParamColor} 0.001 } & {\color{equalParamColor} 0.001 } & {\color{equalParamColor} 0.001 } \\
{\color{equalParamColor} $\varepsilon$ } & {\color{equalParamColor} 1.0 } & {\color{equalParamColor} 1.0 } & {\color{equalParamColor} 1.0 } & {\color{equalParamColor} 1.0 } & {\color{equalParamColor} 1.0 } \\
\hline

\caption{worst settings regarding \textit{final val accuracy avg} for the MNIST revised dataset}
\label{table:final_val_accuracy_avg_worst_mnist_revised}
\end{longtable}


% best parameter values regarding \texttt{final_val_accuracy_avg}
\begin{longtable}{|l|c|c|c|c|c|c|c|c|c|c|c|c|c|c|c|c|c|c|c|}
\hline
\textbf{parameter name} & \multicolumn{6}{c|}{\textbf{parameter values}} & \multicolumn{6}{c|}{\textbf{win ratios in \%}} & \multicolumn{6}{c|}{\textbf{avg. differences in \%}} & \textbf{best value} \\
\hline
neurons per layer & \multicolumn{2}{c:}{(32,)} & \multicolumn{2}{c:}{(50, 10)} & \multicolumn{2}{c|}{(20, 15, 10)} & \multicolumn{2}{c:}{63.8} & \multicolumn{2}{c:}{31.4} & \multicolumn{2}{c|}{4.9} & \multicolumn{2}{c:}{1.091} & \multicolumn{2}{c:}{20.613} & \multicolumn{2}{c|}{27.614} & (32,) \\
activation functions & \multicolumn{3}{c:}{ReLU} & \multicolumn{3}{c|}{sigmoid} & \multicolumn{3}{c:}{73.1} & \multicolumn{3}{c|}{26.9} & \multicolumn{3}{c:}{8.66} & \multicolumn{3}{c|}{20.247} & ReLU \\
last activation function & \multicolumn{3}{c:}{softmax} & \multicolumn{3}{c|}{sigmoid} & \multicolumn{3}{c:}{62.3} & \multicolumn{3}{c|}{37.7} & \multicolumn{3}{c:}{2.423} & \multicolumn{3}{c|}{10.791} & softmax \\
loss function & \multicolumn{3}{c:}{cat-cross} & \multicolumn{3}{c|}{MSE} & \multicolumn{3}{c:}{86.7} & \multicolumn{3}{c|}{13.3} & \multicolumn{3}{c:}{0.401} & \multicolumn{3}{c|}{33.215} & cat-cross \\
number of epochs & \multicolumn{2}{c:}{5} & \multicolumn{2}{c:}{25} & \multicolumn{2}{c|}{50} & \multicolumn{2}{c:}{6.8} & \multicolumn{2}{c:}{19.2} & \multicolumn{2}{c|}{74.0} & \multicolumn{2}{c:}{25.91} & \multicolumn{2}{c:}{9.915} & \multicolumn{2}{c|}{1.916} & 50 \\
batch size & \multicolumn{2}{c:}{100} & \multicolumn{2}{c:}{1000} & \multicolumn{2}{c|}{10000} & \multicolumn{2}{c:}{74.5} & \multicolumn{2}{c:}{19.4} & \multicolumn{2}{c|}{6.0} & \multicolumn{2}{c:}{4.087} & \multicolumn{2}{c:}{21.393} & \multicolumn{2}{c|}{38.937} & 100 \\
optimizer & \multicolumn{3}{c:}{Adam} & \multicolumn{3}{c|}{cAdam} & \multicolumn{3}{c:}{34.6} & \multicolumn{3}{c|}{65.4} & \multicolumn{3}{c:}{7.61} & \multicolumn{3}{c|}{2.37} & cAdam \\
learning rate & \multicolumn{2}{c:}{0.1} & \multicolumn{2}{c:}{0.01} & \multicolumn{2}{c|}{0.001} & \multicolumn{2}{c:}{46.9} & \multicolumn{2}{c:}{35.1} & \multicolumn{2}{c|}{18.1} & \multicolumn{2}{c:}{20.242} & \multicolumn{2}{c:}{16.717} & \multicolumn{2}{c|}{30.161} & \textit{unclear} \\
$\varepsilon$ & \multicolumn{3}{c:}{1.0} & \multicolumn{3}{c|}{$10^{-7}$} & \multicolumn{3}{c:}{16.2} & \multicolumn{3}{c|}{83.8} & \multicolumn{3}{c:}{52.679} & \multicolumn{3}{c|}{7.421} & $10^{-7}$ \\
\hline

\caption{parameter influence regarding \textit{final val accuracy avg} for the MNIST revised dataset}
\label{table:final_val_accuracy_avg_ratios_mnist_revised}
\end{longtable}



%%%%%%%%%%%%%%%%%%%%%%%%%%%%%%%%%%%%%%%%%%%%%%%%%%%%%%%%%%%%%%%%%%%%%%%
\section{ChemRegB with \% differences}
\subsection{training time}
\begin{longtable}{|l|>{\columncolor{bestColumnColor}}l|l|l|l|l|}
\hline
\textbf{parameter name} & \multicolumn{5}{c|}{\textbf{best values}} \\
\hline
\textit{training time avg} &  1.9308 &   1.932 &  1.9436 &  1.9454 &  1.9488 \\
{\color{equalParamColor} neurons per layer } & {\color{equalParamColor} (32,) } & {\color{equalParamColor} (32,) } & {\color{equalParamColor} (32,) } & {\color{equalParamColor} (32,) } & {\color{equalParamColor} (32,) } \\
{\color{equalParamColor} activation functions } & {\color{equalParamColor} sigmoid } & {\color{equalParamColor} sigmoid } & {\color{equalParamColor} sigmoid } & {\color{equalParamColor} sigmoid } & {\color{equalParamColor} sigmoid } \\
{\color{equalParamColor} last activation function } & {\color{equalParamColor} linear } & {\color{equalParamColor} linear } & {\color{equalParamColor} linear } & {\color{equalParamColor} linear } & {\color{equalParamColor} linear } \\
{\color{equalParamColor} loss function } & {\color{equalParamColor} MSE } & {\color{equalParamColor} MSE } & {\color{equalParamColor} MSE } & {\color{equalParamColor} MSE } & {\color{equalParamColor} MSE } \\
{\color{equalParamColor} training data percentage } & {\color{equalParamColor} 1 } & {\color{equalParamColor} 1 } & {\color{equalParamColor} 1 } & {\color{equalParamColor} 1 } & {\color{equalParamColor} 1 } \\
{\color{equalParamColor} number of epochs } & {\color{equalParamColor} 50 } & {\color{equalParamColor} 50 } & {\color{equalParamColor} 50 } & {\color{equalParamColor} 50 } & {\color{equalParamColor} 50 } \\
{\color{equalParamColor} batch size } & {\color{equalParamColor} 10000 } & {\color{equalParamColor} 10000 } & {\color{equalParamColor} 10000 } & {\color{equalParamColor} 10000 } & {\color{equalParamColor} 10000 } \\
{\color{equalParamColor} optimizer } & {\color{equalParamColor} Adam } & {\color{equalParamColor} Adam } & {\color{equalParamColor} Adam } & {\color{equalParamColor} Adam } & {\color{equalParamColor} Adam } \\
learning rate            & 0.01    & 0.1     & 0.001   & 0.01    & 0.1     \\
$\varepsilon$            & $10^{-7}$ & 1       & $10^{-7}$ & 1       & $10^{-7}$ \\
\hline

\caption{best settings regarding \textit{training time avg} for the chemReg Adam dataset}
\label{table:training_time_avg_best_chemreg_adam}
\end{longtable}

\begin{longtable}{|l|l|l|l|l|>{\columncolor{worstColumnColor}}l|}
\hline
\textbf{parameter name} & \multicolumn{5}{c|}{\textbf{worst values}} \\
\hline
\textit{training time avg} &  629.59 &  634.79 &  640.87 &  643.69 &  652.02 \\
{\color{equalParamColor} neurons per layer } & {\color{equalParamColor} (20, 15, 10) } & {\color{equalParamColor} (20, 15, 10) } & {\color{equalParamColor} (20, 15, 10) } & {\color{equalParamColor} (20, 15, 10) } & {\color{equalParamColor} (20, 15, 10) } \\
{\color{equalParamColor} activation functions } & {\color{equalParamColor} sigmoid } & {\color{equalParamColor} sigmoid } & {\color{equalParamColor} sigmoid } & {\color{equalParamColor} sigmoid } & {\color{equalParamColor} sigmoid } \\
{\color{equalParamColor} last activation function } & {\color{equalParamColor} linear } & {\color{equalParamColor} linear } & {\color{equalParamColor} linear } & {\color{equalParamColor} linear } & {\color{equalParamColor} linear } \\
{\color{equalParamColor} loss function } & {\color{equalParamColor} log cosh } & {\color{equalParamColor} log cosh } & {\color{equalParamColor} log cosh } & {\color{equalParamColor} log cosh } & {\color{equalParamColor} log cosh } \\
{\color{equalParamColor} training data percentage } & {\color{equalParamColor} 1 } & {\color{equalParamColor} 1 } & {\color{equalParamColor} 1 } & {\color{equalParamColor} 1 } & {\color{equalParamColor} 1 } \\
{\color{equalParamColor} number of epochs } & {\color{equalParamColor} 500 } & {\color{equalParamColor} 500 } & {\color{equalParamColor} 500 } & {\color{equalParamColor} 500 } & {\color{equalParamColor} 500 } \\
{\color{equalParamColor} batch size } & {\color{equalParamColor} 100 } & {\color{equalParamColor} 100 } & {\color{equalParamColor} 100 } & {\color{equalParamColor} 100 } & {\color{equalParamColor} 100 } \\
{\color{equalParamColor} optimizer } & {\color{equalParamColor} Adam } & {\color{equalParamColor} Adam } & {\color{equalParamColor} Adam } & {\color{equalParamColor} Adam } & {\color{equalParamColor} Adam } \\
learning rate            & 0.1     & 0.001   & 0.1     & 0.01    & 0.01    \\
$\varepsilon$            & 1       & $10^{-7}$ & $10^{-7}$ & 1       & $10^{-7}$ \\
\hline

\caption{worst settings regarding \textit{training time avg} for the chemReg Adam dataset}
\label{table:training_time_avg_worst_chemreg_adam}
\end{longtable}

% best parameter values regarding \texttt{training_time_avg}
\begin{longtable}{|l|c|c|c|c|c|c|c|c|c|c|c|c|c|c|c|c|c|c|c|}
\hline
\textbf{parameter name} & \multicolumn{6}{c|}{\textbf{parameter values}} & \multicolumn{6}{c|}{\textbf{win ratios in \%}} & \multicolumn{6}{c|}{\textbf{avg. differences in \%}} & \textbf{best value} \\
\hline
neurons per layer & \multicolumn{2}{c:}{(40, 20)} & \multicolumn{2}{c:}{(50, 10)} & \multicolumn{2}{c|}{(30, 30, 10)} & \multicolumn{2}{c:}{87.5} & \multicolumn{2}{c:}{12.5} & \multicolumn{2}{c|}{  0} & \multicolumn{2}{c:}{0.151} & \multicolumn{2}{c:}{1.948} & \multicolumn{2}{c|}{11.851} & (40, 20) \\
activation functions & \multicolumn{3}{c:}{ReLU} & \multicolumn{3}{c|}{sigmoid} & \multicolumn{3}{c:}{51.4} & \multicolumn{3}{c|}{48.6} & \multicolumn{3}{c:}{0.63} & \multicolumn{3}{c|}{0.614} & \textit{unclear} \\
loss function & \multicolumn{3}{c:}{MSE} & \multicolumn{3}{c|}{log cosh} & \multicolumn{3}{c:}{87.5} & \multicolumn{3}{c|}{12.5} & \multicolumn{3}{c:}{0.126} & \multicolumn{3}{c|}{2.565} & MSE \\
batch size & \multicolumn{2}{c:}{100} & \multicolumn{2}{c:}{1000} & \multicolumn{2}{c|}{10000} & \multicolumn{2}{c:}{  0} & \multicolumn{2}{c:}{  0} & \multicolumn{2}{c|}{100.0} & \multicolumn{2}{c:}{2448.099} & \multicolumn{2}{c:}{214.332} & \multicolumn{2}{c|}{    0} & 10000 \\
optimizer & \multicolumn{3}{c:}{Adam} & \multicolumn{3}{c|}{cAdam} & \multicolumn{3}{c:}{100.0} & \multicolumn{3}{c|}{  0} & \multicolumn{3}{c:}{    0} & \multicolumn{3}{c|}{47.609} & Adam \\
learning rate & \multicolumn{3}{c:}{0.01} & \multicolumn{3}{c|}{0.001} & \multicolumn{3}{c:}{43.1} & \multicolumn{3}{c|}{56.9} & \multicolumn{3}{c:}{0.527} & \multicolumn{3}{c|}{0.577} & \textit{unclear} \\
\hline

\caption{parameter influence regarding \textit{training time avg} for the chemReg Adam dataset}
\label{table:training_time_avg_ratios_chemreg_adam}
\end{longtable}


\subsection{validation loss}
\begin{longtable}{|l|>{\columncolor{bestColumnColor}}l|l|l|l|l|}
\hline
\textbf{parameter name} & \multicolumn{5}{c|}{\textbf{best values}} \\
\hline
\textit{'final val loss mae avg'} (avg) &  0.1116 & 0.11208 & 0.11416 & 0.11493 & 0.11534 \\
test loss mae avg        & 0.11389 & 0.11298 & 0.11493 & 0.11445 & 0.11539 \\
test loss avg            & 0.012971 & 0.012764 & 0.0065903 & 0.0065354 & 0.0066426 \\
training time avg        & 44.379  & 566.85  & 577.92  & 80.447  & 44.17   \\
neurons per layer        & (50, 10) & (50, 10) & (50, 10) & (30, 30, 10) & (40, 20) \\
{\color{equalParamColor} activation functions } & {\color{equalParamColor} sigmoid } & {\color{equalParamColor} sigmoid } & {\color{equalParamColor} sigmoid } & {\color{equalParamColor} sigmoid } & {\color{equalParamColor} sigmoid } \\
{\color{equalParamColor} last activation function } & {\color{equalParamColor} linear } & {\color{equalParamColor} linear } & {\color{equalParamColor} linear } & {\color{equalParamColor} linear } & {\color{equalParamColor} linear } \\
loss function            & MSE     & MSE     & log cosh & log cosh & log cosh \\
{\color{equalParamColor} training data percentage } & {\color{equalParamColor} 1 } & {\color{equalParamColor} 1 } & {\color{equalParamColor} 1 } & {\color{equalParamColor} 1 } & {\color{equalParamColor} 1 } \\
{\color{equalParamColor} number of epochs } & {\color{equalParamColor} 500 } & {\color{equalParamColor} 500 } & {\color{equalParamColor} 500 } & {\color{equalParamColor} 500 } & {\color{equalParamColor} 500 } \\
batch size               & 1000    & 100     & 100     & 1000    & 1000    \\
optimizer                & Adam    & cAdam   & cAdam   & cAdam   & Adam    \\
learning rate            & 0.01    & 0.001   & 0.001   & 0.01    & 0.01    \\
{\color{equalParamColor} $\varepsilon$ } & {\color{equalParamColor} $10^{-7}$ } & {\color{equalParamColor} $10^{-7}$ } & {\color{equalParamColor} $10^{-7}$ } & {\color{equalParamColor} $10^{-7}$ } & {\color{equalParamColor} $10^{-7}$ } \\
\hline

\caption{best settings regarding \textit{final val loss mae avg} for the chemReg Adam dataset}
\label{table:final_val_loss_mae_avg_best_chemreg_adam}
\end{longtable}

\begin{longtable}{|l|l|l|l|l|>{\columncolor{worstColumnColor}}l|}
\hline
\textbf{parameter name} & \multicolumn{5}{c|}{\textbf{worst values}} \\
\hline
\textit{'final val loss mae avg'} (avg) & 0.19109 & 0.19588 & 0.20277 & 0.21157 & 0.22639 \\
test loss mae avg        & 0.18477 & 0.19365 & 0.19938 & 0.20834 & 0.22996 \\
test loss avg            & 0.034141 & 0.018634 & 0.019747 & 0.043405 & 0.052883 \\
training time avg        & 17.242  & 17.453  & 17.19   & 17.624  & 16.914  \\
neurons per layer        & (50, 10) & (50, 10) & (40, 20) & (30, 30, 10) & (40, 20) \\
{\color{equalParamColor} activation functions } & {\color{equalParamColor} sigmoid } & {\color{equalParamColor} sigmoid } & {\color{equalParamColor} sigmoid } & {\color{equalParamColor} sigmoid } & {\color{equalParamColor} sigmoid } \\
{\color{equalParamColor} last activation function } & {\color{equalParamColor} linear } & {\color{equalParamColor} linear } & {\color{equalParamColor} linear } & {\color{equalParamColor} linear } & {\color{equalParamColor} linear } \\
loss function            & MSE     & log cosh & log cosh & MSE     & MSE     \\
{\color{equalParamColor} training data percentage } & {\color{equalParamColor} 1 } & {\color{equalParamColor} 1 } & {\color{equalParamColor} 1 } & {\color{equalParamColor} 1 } & {\color{equalParamColor} 1 } \\
{\color{equalParamColor} number of epochs } & {\color{equalParamColor} 500 } & {\color{equalParamColor} 500 } & {\color{equalParamColor} 500 } & {\color{equalParamColor} 500 } & {\color{equalParamColor} 500 } \\
{\color{equalParamColor} batch size } & {\color{equalParamColor} 10000 } & {\color{equalParamColor} 10000 } & {\color{equalParamColor} 10000 } & {\color{equalParamColor} 10000 } & {\color{equalParamColor} 10000 } \\
{\color{equalParamColor} optimizer } & {\color{equalParamColor} Adam } & {\color{equalParamColor} Adam } & {\color{equalParamColor} Adam } & {\color{equalParamColor} Adam } & {\color{equalParamColor} Adam } \\
{\color{equalParamColor} learning rate } & {\color{equalParamColor} 0.001 } & {\color{equalParamColor} 0.001 } & {\color{equalParamColor} 0.001 } & {\color{equalParamColor} 0.001 } & {\color{equalParamColor} 0.001 } \\
{\color{equalParamColor} $\varepsilon$ } & {\color{equalParamColor} $10^{-7}$ } & {\color{equalParamColor} $10^{-7}$ } & {\color{equalParamColor} $10^{-7}$ } & {\color{equalParamColor} $10^{-7}$ } & {\color{equalParamColor} $10^{-7}$ } \\
\hline

\caption{worst settings regarding \textit{final val loss mae avg} for the chemReg Adam dataset}
\label{table:final_val_loss_mae_avg_worst_chemreg_adam}
\end{longtable}

% best parameter values regarding \texttt{final_val_loss_mae_avg}
\begin{longtable}{|l|c|c|c|c|c|c|c|c|c|c|c|c|c|c|c|c|c|c|c|}
\hline
\textbf{parameter name} & \multicolumn{6}{c|}{\textbf{parameter values}} & \multicolumn{6}{c|}{\textbf{win ratios in \%}} & \multicolumn{6}{c|}{\textbf{avg. differences}} & \textbf{best value} \\
\hline
neurons per layer & \multicolumn{2}{c:}{(32,)} & \multicolumn{2}{c:}{(50, 10)} & \multicolumn{2}{c|}{(20, 15, 10)} & \multicolumn{2}{c:}{27.3} & \multicolumn{2}{c:}{69.4} & \multicolumn{2}{c|}{3.2} & \multicolumn{2}{c:}{0.015} & \multicolumn{2}{c:}{0.034} & \multicolumn{2}{c|}{0.058} & \textit{unclear} \\
activation functions & \multicolumn{3}{c:}{ReLU} & \multicolumn{3}{c|}{sigmoid} & \multicolumn{3}{c:}{67.0} & \multicolumn{3}{c|}{33.0} & \multicolumn{3}{c:}{0.019} & \multicolumn{3}{c|}{0.065} & ReLU \\
loss function & \multicolumn{3}{c:}{MSE} & \multicolumn{3}{c|}{log cosh} & \multicolumn{3}{c:}{54.9} & \multicolumn{3}{c|}{45.1} & \multicolumn{3}{c:}{0.013} & \multicolumn{3}{c|}{0.025} & MSE \\
number of epochs & \multicolumn{2}{c:}{50} & \multicolumn{2}{c:}{150} & \multicolumn{2}{c|}{500} & \multicolumn{2}{c:}{2.8} & \multicolumn{2}{c:}{5.1} & \multicolumn{2}{c|}{92.1} & \multicolumn{2}{c:}{0.138} & \multicolumn{2}{c:}{0.059} & \multicolumn{2}{c|}{0.006} & 500 \\
batch size & \multicolumn{2}{c:}{100} & \multicolumn{2}{c:}{1000} & \multicolumn{2}{c|}{10000} & \multicolumn{2}{c:}{73.1} & \multicolumn{2}{c:}{17.1} & \multicolumn{2}{c|}{9.7} & \multicolumn{2}{c:}{0.017} & \multicolumn{2}{c:}{0.066} & \multicolumn{2}{c|}{0.226} & 100 \\
learning rate & \multicolumn{2}{c:}{0.1} & \multicolumn{2}{c:}{0.01} & \multicolumn{2}{c|}{0.001} & \multicolumn{2}{c:}{57.9} & \multicolumn{2}{c:}{29.6} & \multicolumn{2}{c|}{12.5} & \multicolumn{2}{c:}{0.041} & \multicolumn{2}{c:}{0.057} & \multicolumn{2}{c|}{0.194} & 0.1 \\
$\varepsilon$ & \multicolumn{3}{c:}{1} & \multicolumn{3}{c|}{$10^{-7}$} & \multicolumn{3}{c:}{16.0} & \multicolumn{3}{c|}{84.0} & \multicolumn{3}{c:}{0.209} & \multicolumn{3}{c|}{0.021} & $10^{-7}$ \\
\hline

\caption{parameter influence regarding \textit{final val loss mae avg} for the chemReg Adam dataset}
\label{table:final_val_loss_mae_avg_ratios_chemreg_adam}
\end{longtable}


\subsection{test loss}
\begin{longtable}{|l|>{\columncolor{bestColumnColor}}l|l|l|l|l|}
\hline
\textbf{parameter name} & \multicolumn{5}{c|}{\textbf{best values}} \\
\hline
\textit{'test loss avg'} (avg) & 0.11298 & 0.11379 & 0.11389 & 0.11409 & 0.11445 \\
test loss mae avg        & 0.11298 & 0.11379 & 0.11389 & 0.11409 & 0.11445 \\
final val loss avg       & 0.012562 & 0.013544 & 0.012455 & 0.0066663 & 0.0065895 \\
training time avg        & 566.85  & 43.583  & 44.379  & 576.57  & 80.447  \\
neurons per layer        & (50, 10) & (40, 20) & (50, 10) & (40, 20) & (30, 30, 10) \\
{\color{equalParamColor} activation functions } & {\color{equalParamColor} sigmoid } & {\color{equalParamColor} sigmoid } & {\color{equalParamColor} sigmoid } & {\color{equalParamColor} sigmoid } & {\color{equalParamColor} sigmoid } \\
{\color{equalParamColor} last activation function } & {\color{equalParamColor} linear } & {\color{equalParamColor} linear } & {\color{equalParamColor} linear } & {\color{equalParamColor} linear } & {\color{equalParamColor} linear } \\
loss function            & MSE     & MSE     & MSE     & log cosh & log cosh \\
{\color{equalParamColor} training data percentage } & {\color{equalParamColor} 1 } & {\color{equalParamColor} 1 } & {\color{equalParamColor} 1 } & {\color{equalParamColor} 1 } & {\color{equalParamColor} 1 } \\
{\color{equalParamColor} number of epochs } & {\color{equalParamColor} 500 } & {\color{equalParamColor} 500 } & {\color{equalParamColor} 500 } & {\color{equalParamColor} 500 } & {\color{equalParamColor} 500 } \\
batch size               & 100     & 1000    & 1000    & 100     & 1000    \\
optimizer                & cAdam   & Adam    & Adam    & cAdam   & cAdam   \\
learning rate            & 0.001   & 0.01    & 0.01    & 0.001   & 0.01    \\
{\color{equalParamColor} $\varepsilon$ } & {\color{equalParamColor} $10^{-7}$ } & {\color{equalParamColor} $10^{-7}$ } & {\color{equalParamColor} $10^{-7}$ } & {\color{equalParamColor} $10^{-7}$ } & {\color{equalParamColor} $10^{-7}$ } \\
\hline

\caption{best settings regarding \textit{test loss avg} for the chemReg Adam dataset}
\label{table:test_loss_avg_best_chemreg_adam}
\end{longtable}

\begin{longtable}{|l|l|l|l|l|>{\columncolor{worstColumnColor}}l|}
\hline
\textbf{parameter name} & \multicolumn{5}{c|}{\textbf{worst values}} \\
\hline
\textit{MAE}             &  1.3636 &    1.38 &  1.4869 &   1.531 &  1.5561 \\
test loss avg            & 0.75386 & 1.915   & 2.4475  & 0.90854 & 0.90961 \\
final val loss avg       & 0.78188 & 1.9706  & 2.5777  & 0.93786 & 0.94109 \\
final val loss std       & 0.4144  & 0.24678 & 1.5986  & 0.4412  & 0.18681 \\
final val loss min       & 0.41245 & 1.67    & 0.94529 & 0.2746  & 0.65788 \\
final val loss max       & 1.5601  & 2.3295  & 5.2752  & 1.6236  & 1.1796  \\
training time avg        & 1.9624  & 1.9914  & 2.0139  & 2.0641  & 2.0032  \\
neurons per layer        & (32,)   & (32,)   & (50, 10) & (50, 10) & (32,)   \\
activation functions     & sigmoid & ReLU    & sigmoid & sigmoid & ReLU    \\
{\color{equalParamColor} last activation function } & {\color{equalParamColor} linear } & {\color{equalParamColor} linear } & {\color{equalParamColor} linear } & {\color{equalParamColor} linear } & {\color{equalParamColor} linear } \\
loss function            & log cosh & MSE     & MSE     & log cosh & log cosh \\
{\color{equalParamColor} training data percentage } & {\color{equalParamColor} 1 } & {\color{equalParamColor} 1 } & {\color{equalParamColor} 1 } & {\color{equalParamColor} 1 } & {\color{equalParamColor} 1 } \\
{\color{equalParamColor} number of epochs } & {\color{equalParamColor} 50 } & {\color{equalParamColor} 50 } & {\color{equalParamColor} 50 } & {\color{equalParamColor} 50 } & {\color{equalParamColor} 50 } \\
{\color{equalParamColor} batch size } & {\color{equalParamColor} 10000 } & {\color{equalParamColor} 10000 } & {\color{equalParamColor} 10000 } & {\color{equalParamColor} 10000 } & {\color{equalParamColor} 10000 } \\
{\color{equalParamColor} optimizer } & {\color{equalParamColor} Adam } & {\color{equalParamColor} Adam } & {\color{equalParamColor} Adam } & {\color{equalParamColor} Adam } & {\color{equalParamColor} Adam } \\
{\color{equalParamColor} learning rate } & {\color{equalParamColor} 0.001 } & {\color{equalParamColor} 0.001 } & {\color{equalParamColor} 0.001 } & {\color{equalParamColor} 0.001 } & {\color{equalParamColor} 0.001 } \\
{\color{equalParamColor} $\varepsilon$ } & {\color{equalParamColor} 1 } & {\color{equalParamColor} 1 } & {\color{equalParamColor} 1 } & {\color{equalParamColor} 1 } & {\color{equalParamColor} 1 } \\
\hline

\caption{worst settings regarding \textit{test loss avg} for the chemReg Adam dataset}
\label{table:test_loss_avg_worst_chemreg_adam}
\end{longtable}

% best parameter values regarding \texttt{test_loss_avg}
\begin{longtable}{|l|c|c|c|c|c|c|c|c|c|c|c|c|c|c|c|c|c|c|c|}
\hline
\textbf{parameter name} & \multicolumn{6}{c|}{\textbf{parameter values}} & \multicolumn{6}{c|}{\textbf{win ratios in \%}} & \multicolumn{6}{c|}{\textbf{avg. differences in \%}} & \textbf{best value} \\
\hline
neurons per layer & \multicolumn{2}{c:}{(40, 20)} & \multicolumn{2}{c:}{(50, 10)} & \multicolumn{2}{c|}{(30, 30, 10)} & \multicolumn{2}{c:}{27.1} & \multicolumn{2}{c:}{50.0} & \multicolumn{2}{c|}{22.9} & \multicolumn{2}{c:}{3.242} & \multicolumn{2}{c:}{1.677} & \multicolumn{2}{c|}{5.055} & (50, 10) \\
activation functions & \multicolumn{3}{c:}{ReLU} & \multicolumn{3}{c|}{sigmoid} & \multicolumn{3}{c:}{55.6} & \multicolumn{3}{c|}{44.4} & \multicolumn{3}{c:}{2.313} & \multicolumn{3}{c|}{4.219} & ReLU \\
loss function & \multicolumn{3}{c:}{MSE} & \multicolumn{3}{c|}{log cosh} & \multicolumn{3}{c:}{37.5} & \multicolumn{3}{c|}{62.5} & \multicolumn{3}{c:}{2.93} & \multicolumn{3}{c|}{1.147} & log cosh \\
batch size & \multicolumn{2}{c:}{100} & \multicolumn{2}{c:}{1000} & \multicolumn{2}{c|}{10000} & \multicolumn{2}{c:}{50.0} & \multicolumn{2}{c:}{47.9} & \multicolumn{2}{c|}{2.1} & \multicolumn{2}{c:}{4.201} & \multicolumn{2}{c:}{4.151} & \multicolumn{2}{c|}{25.616} & \textit{unclear} \\
optimizer & \multicolumn{3}{c:}{Adam} & \multicolumn{3}{c|}{cAdam} & \multicolumn{3}{c:}{45.8} & \multicolumn{3}{c|}{54.2} & \multicolumn{3}{c:}{3.551} & \multicolumn{3}{c|}{1.499} & cAdam \\
learning rate & \multicolumn{3}{c:}{0.01} & \multicolumn{3}{c|}{0.001} & \multicolumn{3}{c:}{65.3} & \multicolumn{3}{c|}{34.7} & \multicolumn{3}{c:}{3.06} & \multicolumn{3}{c|}{10.756} & 0.01 \\
\hline

\caption{parameter influence regarding \textit{test loss avg} for the chemReg Adam dataset}
\label{table:test_loss_avg_ratios_chemreg_adam}
\end{longtable}


\end{landscape}
\end{document}